\documentclass[11pt,]{article}
\usepackage{lmodern}
\usepackage{amssymb,amsmath}
\usepackage{ifxetex,ifluatex}
\usepackage{fixltx2e} % provides \textsubscript
\ifnum 0\ifxetex 1\fi\ifluatex 1\fi=0 % if pdftex
  \usepackage[T1]{fontenc}
  \usepackage[utf8]{inputenc}
\else % if luatex or xelatex
  \ifxetex
    \usepackage{mathspec}
  \else
    \usepackage{fontspec}
  \fi
  \defaultfontfeatures{Ligatures=TeX,Scale=MatchLowercase}
\fi
% use upquote if available, for straight quotes in verbatim environments
\IfFileExists{upquote.sty}{\usepackage{upquote}}{}
% use microtype if available
\IfFileExists{microtype.sty}{%
\usepackage{microtype}
\UseMicrotypeSet[protrusion]{basicmath} % disable protrusion for tt fonts
}{}
\usepackage[margin=1in]{geometry}
\usepackage{hyperref}
\hypersetup{unicode=true,
            pdftitle={Appendix S1. Instructions for retrieving and archiving the environmental covariates.},
            pdfborder={0 0 0},
            breaklinks=true}
\urlstyle{same}  % don't use monospace font for urls
\usepackage{color}
\usepackage{fancyvrb}
\newcommand{\VerbBar}{|}
\newcommand{\VERB}{\Verb[commandchars=\\\{\}]}
\DefineVerbatimEnvironment{Highlighting}{Verbatim}{commandchars=\\\{\}}
% Add ',fontsize=\small' for more characters per line
\newenvironment{Shaded}{}{}
\newcommand{\KeywordTok}[1]{\textcolor[rgb]{0.00,0.00,1.00}{#1}}
\newcommand{\DataTypeTok}[1]{#1}
\newcommand{\DecValTok}[1]{#1}
\newcommand{\BaseNTok}[1]{#1}
\newcommand{\FloatTok}[1]{#1}
\newcommand{\ConstantTok}[1]{#1}
\newcommand{\CharTok}[1]{\textcolor[rgb]{0.00,0.50,0.50}{#1}}
\newcommand{\SpecialCharTok}[1]{\textcolor[rgb]{0.00,0.50,0.50}{#1}}
\newcommand{\StringTok}[1]{\textcolor[rgb]{0.00,0.50,0.50}{#1}}
\newcommand{\VerbatimStringTok}[1]{\textcolor[rgb]{0.00,0.50,0.50}{#1}}
\newcommand{\SpecialStringTok}[1]{\textcolor[rgb]{0.00,0.50,0.50}{#1}}
\newcommand{\ImportTok}[1]{#1}
\newcommand{\CommentTok}[1]{\textcolor[rgb]{0.00,0.50,0.00}{#1}}
\newcommand{\DocumentationTok}[1]{\textcolor[rgb]{0.00,0.50,0.00}{#1}}
\newcommand{\AnnotationTok}[1]{\textcolor[rgb]{0.00,0.50,0.00}{#1}}
\newcommand{\CommentVarTok}[1]{\textcolor[rgb]{0.00,0.50,0.00}{#1}}
\newcommand{\OtherTok}[1]{\textcolor[rgb]{1.00,0.25,0.00}{#1}}
\newcommand{\FunctionTok}[1]{#1}
\newcommand{\VariableTok}[1]{#1}
\newcommand{\ControlFlowTok}[1]{\textcolor[rgb]{0.00,0.00,1.00}{#1}}
\newcommand{\OperatorTok}[1]{#1}
\newcommand{\BuiltInTok}[1]{#1}
\newcommand{\ExtensionTok}[1]{#1}
\newcommand{\PreprocessorTok}[1]{\textcolor[rgb]{1.00,0.25,0.00}{#1}}
\newcommand{\AttributeTok}[1]{#1}
\newcommand{\RegionMarkerTok}[1]{#1}
\newcommand{\InformationTok}[1]{\textcolor[rgb]{0.00,0.50,0.00}{#1}}
\newcommand{\WarningTok}[1]{\textcolor[rgb]{0.00,0.50,0.00}{\textbf{#1}}}
\newcommand{\AlertTok}[1]{\textcolor[rgb]{1.00,0.00,0.00}{#1}}
\newcommand{\ErrorTok}[1]{\textcolor[rgb]{1.00,0.00,0.00}{\textbf{#1}}}
\newcommand{\NormalTok}[1]{#1}
\usepackage{graphicx,grffile}
\makeatletter
\def\maxwidth{\ifdim\Gin@nat@width>\linewidth\linewidth\else\Gin@nat@width\fi}
\def\maxheight{\ifdim\Gin@nat@height>\textheight\textheight\else\Gin@nat@height\fi}
\makeatother
% Scale images if necessary, so that they will not overflow the page
% margins by default, and it is still possible to overwrite the defaults
% using explicit options in \includegraphics[width, height, ...]{}
\setkeys{Gin}{width=\maxwidth,height=\maxheight,keepaspectratio}
\IfFileExists{parskip.sty}{%
\usepackage{parskip}
}{% else
\setlength{\parindent}{0pt}
\setlength{\parskip}{6pt plus 2pt minus 1pt}
}
\setlength{\emergencystretch}{3em}  % prevent overfull lines
\providecommand{\tightlist}{%
  \setlength{\itemsep}{0pt}\setlength{\parskip}{0pt}}
\setcounter{secnumdepth}{5}
% Redefines (sub)paragraphs to behave more like sections
\ifx\paragraph\undefined\else
\let\oldparagraph\paragraph
\renewcommand{\paragraph}[1]{\oldparagraph{#1}\mbox{}}
\fi
\ifx\subparagraph\undefined\else
\let\oldsubparagraph\subparagraph
\renewcommand{\subparagraph}[1]{\oldsubparagraph{#1}\mbox{}}
\fi

%%% Use protect on footnotes to avoid problems with footnotes in titles
\let\rmarkdownfootnote\footnote%
\def\footnote{\protect\rmarkdownfootnote}

%%% Change title format to be more compact
\usepackage{titling}

% Create subtitle command for use in maketitle
\newcommand{\subtitle}[1]{
  \posttitle{
    \begin{center}\large#1\end{center}
    }
}

\setlength{\droptitle}{-2em}
  \title{Appendix S1. Instructions for retrieving and archiving the environmental
covariates.}
  \pretitle{\vspace{\droptitle}\centering\huge}
  \posttitle{\par}
\subtitle{Retrospective analysis of Skagit River Chum salmon productivity}
  \author{}
  \preauthor{}\postauthor{}
  \date{}
  \predate{}\postdate{}


\begin{document}
\maketitle

{
\setcounter{tocdepth}{2}
\tableofcontents
}
\vspace{0.2in}

This is version 0.20.03.03.

\section{Background}\label{background}

This appendix describes how to retrieve the environmental covariates
used in our analyses. After reading in the raw data, summarizing them
(if necessary), and trimming them to the appropriate time frame, they
table of covariates is written to a \texttt{.csv} file.

All of the analyses require the \href{https://cran.r-project.org/}{R
software} (v3.4.3 or later) for data retrieval and processing. We also
need the \textbf{readr} and \textbf{here} packages, which are not
included with the base installation of \textbf{R}.

\begin{Shaded}
\begin{Highlighting}[]
\ControlFlowTok{if}\NormalTok{(}\OperatorTok{!}\KeywordTok{require}\NormalTok{(}\StringTok{"readr"}\NormalTok{)) \{}
  \KeywordTok{install.packages}\NormalTok{(}\StringTok{"readr"}\NormalTok{)}
  \KeywordTok{library}\NormalTok{(}\StringTok{"readr"}\NormalTok{)}
\NormalTok{\}}
\ControlFlowTok{if}\NormalTok{(}\OperatorTok{!}\KeywordTok{require}\NormalTok{(}\StringTok{"here"}\NormalTok{)) \{}
  \KeywordTok{install.packages}\NormalTok{(}\StringTok{"here"}\NormalTok{)}
  \KeywordTok{library}\NormalTok{(}\StringTok{"here"}\NormalTok{)}
\NormalTok{\}}
\NormalTok{## set data dir}
\NormalTok{datadir <-}\StringTok{ }\KeywordTok{here}\NormalTok{(}\StringTok{"data"}\NormalTok{)}
\end{Highlighting}
\end{Shaded}

\section{User inputs}\label{user-inputs}

We begin by supplying values for the following parameters, which we use
for trimming and lagging the covariates to the appropriate years.

\begin{Shaded}
\begin{Highlighting}[]
\NormalTok{## first & last years of fish data}
\NormalTok{yr_frst <-}\StringTok{ }\DecValTok{1980}
\NormalTok{yr_last <-}\StringTok{ }\DecValTok{2018}

\NormalTok{## min & max adult ages (years)}
\NormalTok{age_min <-}\StringTok{ }\DecValTok{3}
\NormalTok{age_max <-}\StringTok{ }\DecValTok{5}

\NormalTok{## time lags (years) for covariates}
\NormalTok{flow_lag <-}\StringTok{ }\DecValTok{1}
\NormalTok{marine_lag <-}\StringTok{ }\DecValTok{1}


\NormalTok{## number of years for run forecasts}
\NormalTok{n_fore <-}\StringTok{ }\DecValTok{0}
\end{Highlighting}
\end{Shaded}

\section{Retrieve covariates}\label{retrieve-covariates}

Our analysis investigates 3 covariates as possible drivers of the
population's instrinic growth rate:

\begin{enumerate}
\def\labelenumi{\arabic{enumi}.}
\tightlist
\item
  Maximum river discharge in winter;
\item
  Minimum river discharge in summer;
\item
  North Pacific Gyre Oscillation;
\item
  Hatchery SAR
\item
  Whidbey basin pink salmon escapement
\end{enumerate}

\subsection{River discharge}\label{river-discharge}

We begin by getting the daily flow data from the US Geological Service
\href{http://waterdata.usgs.gov/nwis}{National Water Information
System}. We will use the direct link to the gage data from the Skagit
River near Mount Vernon, WA (\#12200500), beginning with the first year
of fish data.

\begin{Shaded}
\begin{Highlighting}[]
\NormalTok{## flow gage ID}
\NormalTok{flow_site <-}\StringTok{ }\DecValTok{12200500}
\NormalTok{## get URL for flow data from USGS}
\NormalTok{flow_url <-}\StringTok{ }\KeywordTok{paste0}\NormalTok{(}\StringTok{"https://waterdata.usgs.gov/nwis/dv"}\NormalTok{,}
                   \StringTok{"?cb_00060=on"}\NormalTok{,}
                   \StringTok{"&format=rdb"}\NormalTok{,}
                   \StringTok{"&site_no="}\NormalTok{,flow_site,}
                   \StringTok{"&begin_date="}\NormalTok{,yr_frst,}\StringTok{"-01-01"}\NormalTok{,}
                   \StringTok{"&end_date="}\NormalTok{,yr_last,}\StringTok{"-12-31"}\NormalTok{)}
\end{Highlighting}
\end{Shaded}

Next we retrieve the raw data file and print its metadata.

\begin{Shaded}
\begin{Highlighting}[]
\NormalTok{## raw flow data from USGS}
\NormalTok{flow_raw <-}\StringTok{ }\KeywordTok{read_lines}\NormalTok{(flow_url)}
\NormalTok{## lines with metadata}
\NormalTok{hdr_flow <-}\StringTok{ }\KeywordTok{which}\NormalTok{(}\KeywordTok{lapply}\NormalTok{(flow_raw, grep, }\DataTypeTok{pattern =} \StringTok{"}\CharTok{\textbackslash{}\textbackslash{}}\StringTok{#"}\NormalTok{)}\OperatorTok{==}\DecValTok{1}\NormalTok{, }\DataTypeTok{arr.ind =} \OtherTok{TRUE}\NormalTok{)}
\NormalTok{## print flow metadata}
\KeywordTok{print}\NormalTok{(flow_raw[hdr_flow], }\DataTypeTok{quote =} \OtherTok{FALSE}\NormalTok{)}
\end{Highlighting}
\end{Shaded}

\begin{verbatim}
##  [1] # ---------------------------------- WARNING ----------------------------------------     
##  [2] # Some of the data that you have obtained from this U.S. Geological Survey database       
##  [3] # may not have received Director's approval. Any such data values are qualified           
##  [4] # as provisional and are subject to revision. Provisional data are released on the        
##  [5] # condition that neither the USGS nor the United States Government may be held liable     
##  [6] # for any damages resulting from its use.                                                 
##  [7] #                                                                                         
##  [8] # Additional info: https://help.waterdata.usgs.gov/policies/provisional-data-statement    
##  [9] #                                                                                         
## [10] # File-format description:  https://help.waterdata.usgs.gov/faq/about-tab-delimited-output
## [11] # Automated-retrieval info: https://help.waterdata.usgs.gov/faq/automated-retrievals      
## [12] #                                                                                         
## [13] # Contact:   gs-w_support_nwisweb@usgs.gov                                                
## [14] # retrieved: 2019-07-31 16:34:58 EDT       (vaww02)                                       
## [15] #                                                                                         
## [16] # Data for the following 1 site(s) are contained in this file                             
## [17] #    USGS 12200500 SKAGIT RIVER NEAR MOUNT VERNON, WA                                     
## [18] # -----------------------------------------------------------------------------------     
## [19] #                                                                                         
## [20] # Data provided for site 12200500                                                         
## [21] #            TS   parameter     statistic     Description                                 
## [22] #        149429       00060     00003     Discharge, cubic feet per second (Mean)         
## [23] #                                                                                         
## [24] # Data-value qualification codes included in this output:                                 
## [25] #     A  Approved for publication -- Processing and review completed.                     
## [26] #     e  Value has been estimated.                                                        
## [27] #
\end{verbatim}

Lastly, we extract the actual flow data for the years of interest and
inspect the file contents.

\begin{Shaded}
\begin{Highlighting}[]
\NormalTok{## flow data for years of interest}
\NormalTok{dat_flow <-}\StringTok{  }\KeywordTok{read_tsv}\NormalTok{(flow_url,}
                      \DataTypeTok{col_names =} \OtherTok{FALSE}\NormalTok{,}
                      \DataTypeTok{col_types =} \StringTok{"ciDdc"}\NormalTok{,}
                      \DataTypeTok{skip =} \KeywordTok{max}\NormalTok{(hdr_flow)}\OperatorTok{+}\DecValTok{2}\NormalTok{)}
\KeywordTok{colnames}\NormalTok{(dat_flow) <-}\StringTok{ }\KeywordTok{unlist}\NormalTok{(}\KeywordTok{strsplit}\NormalTok{(}\KeywordTok{tolower}\NormalTok{(flow_raw[}\KeywordTok{max}\NormalTok{(hdr_flow)}\OperatorTok{+}\DecValTok{1}\NormalTok{]),}
                                      \DataTypeTok{split =} \StringTok{"}\CharTok{\textbackslash{}\textbackslash{}}\StringTok{s+"}\NormalTok{))}
\KeywordTok{head}\NormalTok{(dat_flow)}
\end{Highlighting}
\end{Shaded}

\begin{verbatim}
## # A tibble: 6 x 5
##   agency_cd  site_no   datetime `149429_00060_00003` `149429_00060_00003_cd`
##       <chr>    <int>     <date>                <dbl>                   <chr>
## 1      USGS 12200500 1980-01-01                16500                       A
## 2      USGS 12200500 1980-01-02                16700                       A
## 3      USGS 12200500 1980-01-03                17700                       A
## 4      USGS 12200500 1980-01-04                16400                       A
## 5      USGS 12200500 1980-01-05                15800                       A
## 6      USGS 12200500 1980-01-06                15500                       A
\end{verbatim}

We only need the 3rd and 4th columns, which contain the date
(\texttt{datetime}) and daily flow measurements
(\texttt{149429\_00060\_00003}). We will rename them to \texttt{date}
and \texttt{flow}, respectively, and convert the flow units from ``cubic
feet per second'' to ``cubic meters per second''.

\begin{Shaded}
\begin{Highlighting}[]
\NormalTok{## keep only relevant columns}
\NormalTok{dat_flow <-}\StringTok{ }\NormalTok{dat_flow[}\KeywordTok{c}\NormalTok{(}\StringTok{"datetime"}\NormalTok{, }\KeywordTok{grep}\NormalTok{(}\StringTok{"[0-9]$"}\NormalTok{, }\KeywordTok{colnames}\NormalTok{(dat_flow), }\DataTypeTok{value =} \OtherTok{TRUE}\NormalTok{))]}
\NormalTok{## nicer column names}
\KeywordTok{colnames}\NormalTok{(dat_flow) <-}\StringTok{ }\KeywordTok{c}\NormalTok{(}\StringTok{"date"}\NormalTok{,}\StringTok{"flow"}\NormalTok{)}
\NormalTok{## convert cubic feet to cubic meters}
\NormalTok{dat_flow}\OperatorTok{$}\NormalTok{flow <-}\StringTok{ }\NormalTok{dat_flow}\OperatorTok{$}\NormalTok{flow }\OperatorTok{/}\StringTok{ }\FloatTok{35.3147}
\NormalTok{## flow by year & month}
\NormalTok{dat_flow}\OperatorTok{$}\NormalTok{year <-}\StringTok{ }\KeywordTok{as.integer}\NormalTok{(}\KeywordTok{format}\NormalTok{(dat_flow}\OperatorTok{$}\NormalTok{date,}\StringTok{"%Y"}\NormalTok{))}
\NormalTok{dat_flow}\OperatorTok{$}\NormalTok{month <-}\StringTok{ }\KeywordTok{as.integer}\NormalTok{(}\KeywordTok{format}\NormalTok{(dat_flow}\OperatorTok{$}\NormalTok{date,}\StringTok{"%m"}\NormalTok{))}
\NormalTok{dat_flow <-}\StringTok{ }\NormalTok{dat_flow[,}\KeywordTok{c}\NormalTok{(}\StringTok{"year"}\NormalTok{,}\StringTok{"month"}\NormalTok{,}\StringTok{"flow"}\NormalTok{)]}
\end{Highlighting}
\end{Shaded}

\subsubsection{Winter maximum}\label{winter-maximum}

We are interested in the maximum of the daily peak flows from November
through March during the first year that juveniles are rearing in
streams. This means we need to combine flow values from the fall of year
\(t\) with those in the winter and spring of year \(t+1\). We also need
to shift the flow data forward by 1 year so they align with the juvenile
life stage. Therefore, the flow time series will begin in 1980 and end
in 2016.

\begin{Shaded}
\begin{Highlighting}[]
\NormalTok{## autumn flows in year t}
\NormalTok{flow_aut <-}\StringTok{ }\KeywordTok{subset}\NormalTok{(dat_flow, (month}\OperatorTok{>=}\DecValTok{11} \OperatorTok{&}\StringTok{ }\NormalTok{month}\OperatorTok{<=}\DecValTok{12}\NormalTok{)}
                   \OperatorTok{&}\StringTok{ }\NormalTok{year }\OperatorTok{>=}\StringTok{ }\NormalTok{yr_frst }\OperatorTok{&}\StringTok{ }\NormalTok{year }\OperatorTok{<=}\StringTok{ }\NormalTok{yr_last}\OperatorTok{-}\NormalTok{age_min}\OperatorTok{+}\NormalTok{n_fore)}
\NormalTok{## spring flows in year t+1}
\NormalTok{flow_spr <-}\StringTok{ }\KeywordTok{subset}\NormalTok{(dat_flow,}
\NormalTok{                   (month}\OperatorTok{>=}\DecValTok{1} \OperatorTok{&}\StringTok{ }\NormalTok{month}\OperatorTok{<=}\DecValTok{3}\NormalTok{)}
                   \OperatorTok{&}\StringTok{ }\NormalTok{year }\OperatorTok{>=}\StringTok{ }\NormalTok{yr_frst}\OperatorTok{+}\NormalTok{flow_lag}
                   \OperatorTok{&}\StringTok{ }\NormalTok{year }\OperatorTok{<=}\StringTok{ }\NormalTok{yr_last}\OperatorTok{-}\NormalTok{age_min}\OperatorTok{+}\NormalTok{n_fore}\OperatorTok{+}\NormalTok{flow_lag)}
\NormalTok{## change spr year index to match aut}
\NormalTok{flow_spr[,}\StringTok{"year"}\NormalTok{] <-}\StringTok{ }\NormalTok{flow_spr[,}\StringTok{"year"}\NormalTok{] }\OperatorTok{-}\StringTok{ }\NormalTok{flow_lag}
\NormalTok{## combined flows indexed to brood year & calculate max flow}
\CommentTok{#dat_flow_wtr <- aggregate(flow ~ year, data = rbind(flow_aut,flow_spr), mean)}
\NormalTok{dat_flow_wtr <-}\StringTok{ }\KeywordTok{aggregate}\NormalTok{(flow }\OperatorTok{~}\StringTok{ }\NormalTok{year, }\DataTypeTok{data =} \KeywordTok{rbind}\NormalTok{(flow_aut,flow_spr), max)}
\NormalTok{dat_flow_wtr[,}\StringTok{"flow"}\NormalTok{] <-}\StringTok{ }\KeywordTok{round}\NormalTok{(dat_flow_wtr[,}\StringTok{"flow"}\NormalTok{], }\DecValTok{1}\NormalTok{) }
\NormalTok{## change year index to brood year}
\NormalTok{dat_flow_wtr[,}\StringTok{"year"}\NormalTok{] <-}\StringTok{ }\NormalTok{dat_flow_wtr[,}\StringTok{"year"}\NormalTok{] }
\NormalTok{## for plotting purpose later}
\KeywordTok{colnames}\NormalTok{(dat_flow_wtr)[}\DecValTok{2}\NormalTok{] <-}\StringTok{ "flow_wtr"}
\KeywordTok{print}\NormalTok{(dat_flow_wtr)}
\end{Highlighting}
\end{Shaded}

\begin{verbatim}
##    year flow_wtr
## 1  1980   2860.0
## 2  1981   1526.3
## 3  1982   1789.6
## 4  1983   2302.2
## 5  1984    775.9
## 6  1985   1795.3
## 7  1986   1843.4
## 8  1987    909.0
## 9  1988   1336.6
## 10 1989   2497.5
## 11 1990   4021.0
## 12 1991   1135.5
## 13 1992    781.5
## 14 1993   1030.7
## 15 1994   1577.2
## 16 1995   3737.8
## 17 1996   2089.8
## 18 1997   1008.1
## 19 1998   1469.6
## 20 1999   2174.7
## 21 2000    546.5
## 22 2001   2086.9
## 23 2002   1500.8
## 24 2003   1826.4
## 25 2004   1891.6
## 26 2005   1625.4
## 27 2006   3539.6
## 28 2007   2019.0
## 29 2008   2064.3
## 30 2009   1868.9
## 31 2010   2364.5
## 32 2011   1619.7
## 33 2012   1141.2
## 34 2013   1381.9
## 35 2014   2418.3
## 36 2015   2070.0
\end{verbatim}

\subsubsection{Spring max}\label{spring-max}

Retrieving the flow juveniles would experience during their first spring
and early summer rearing in Skagit Bay (April through June) is
straightforward.

\begin{Shaded}
\begin{Highlighting}[]
\NormalTok{## spring flows in year t}
\NormalTok{flow_spr<-}\StringTok{ }\KeywordTok{subset}\NormalTok{(dat_flow, (month}\OperatorTok{>=}\DecValTok{3} \OperatorTok{&}\StringTok{ }\NormalTok{month}\OperatorTok{<=}\DecValTok{6}\NormalTok{)}
                   \OperatorTok{&}\StringTok{ }\NormalTok{year }\OperatorTok{>=}\StringTok{ }\NormalTok{yr_frst}\OperatorTok{+}\NormalTok{flow_lag}
                   \OperatorTok{&}\StringTok{ }\NormalTok{year }\OperatorTok{<=}\StringTok{ }\NormalTok{yr_last}\OperatorTok{-}\NormalTok{age_min}\OperatorTok{+}\NormalTok{n_fore}\OperatorTok{+}\NormalTok{flow_lag)}
\NormalTok{## change year index to brood year}
\NormalTok{flow_spr[,}\StringTok{"year"}\NormalTok{] <-}\StringTok{ }\NormalTok{flow_spr[,}\StringTok{"year"}\NormalTok{] }\OperatorTok{-}\StringTok{ }\NormalTok{flow_lag}
\NormalTok{## combined flows indexed to brood year & calculate average flow}
\NormalTok{dat_flow_spr <-}\StringTok{ }\KeywordTok{aggregate}\NormalTok{(flow }\OperatorTok{~}\StringTok{ }\NormalTok{year, }\DataTypeTok{data =}\NormalTok{ flow_spr, max)}
\NormalTok{dat_flow_spr <-}\StringTok{ }\KeywordTok{round}\NormalTok{(dat_flow_spr, }\DecValTok{2}\NormalTok{)}
\NormalTok{## for plotting purpose later}
\KeywordTok{colnames}\NormalTok{(dat_flow_spr)[}\DecValTok{2}\NormalTok{] <-}\StringTok{ "flow_spr"}
\KeywordTok{print}\NormalTok{(dat_flow_spr)}
\end{Highlighting}
\end{Shaded}

\begin{verbatim}
##    year flow_spr
## 1  1980   917.46
## 2  1981  1347.88
## 3  1982   846.67
## 4  1983   923.13
## 5  1984  1175.15
## 6  1985  1030.73
## 7  1986  1118.51
## 8  1987  1095.86
## 9  1988   846.67
## 10 1989   841.01
## 11 1990   869.33
## 12 1991  1160.99
## 13 1992  1022.24
## 14 1993  1030.73
## 15 1994   736.24
## 16 1995   906.14
## 17 1996  2089.78
## 18 1997   682.44
## 19 1998  1197.80
## 20 1999  1033.56
## 21 2000   744.73
## 22 2001  1452.65
## 23 2002  1279.92
## 24 2003   753.23
## 25 2004   569.17
## 26 2005  1095.86
## 27 2006  2101.11
## 28 2007  1735.82
## 29 2008   886.32
## 30 2009   843.84
## 31 2010  1177.98
## 32 2011  1628.22
## 33 2012  1325.23
## 34 2013  1381.86
## 35 2014   673.94
## 36 2015   761.72
\end{verbatim}

\subsection{North Pacific Gyre
Oscillation}\label{north-pacific-gyre-oscillation}

We used the monthly NPGO data provided by Emanuele Di Lorenzo of the
Georgia Institute of Technology, which are available
\href{http://www.o3d.org/npgo/npgo.php}{here}. We begin by downloading
the raw NPGO data and viewing the metadata.

\begin{Shaded}
\begin{Highlighting}[]
\NormalTok{## URL for NPGO data}
\NormalTok{url_NPGO <-}\StringTok{ "http://www.o3d.org/npgo/npgo.php"}
\NormalTok{## raw NPGO data }
\NormalTok{NPGO_raw <-}\StringTok{ }\KeywordTok{read_lines}\NormalTok{(url_NPGO)}
\NormalTok{## line with data headers}
\NormalTok{hdr_NPGO <-}\StringTok{ }\KeywordTok{which}\NormalTok{(}\KeywordTok{lapply}\NormalTok{(NPGO_raw,grep,}\DataTypeTok{pattern=}\StringTok{"YEAR"}\NormalTok{)}\OperatorTok{==}\DecValTok{1}\NormalTok{, }\DataTypeTok{arr.ind =} \OtherTok{TRUE}\NormalTok{)}
\NormalTok{## print PDO metadata}
\KeywordTok{print}\NormalTok{(NPGO_raw[}\KeywordTok{seq}\NormalTok{(hdr_NPGO)],}\DataTypeTok{quote =} \OtherTok{FALSE}\NormalTok{)}
\end{Highlighting}
\end{Shaded}

\begin{verbatim}
##  [1]                                                                                                     
##  [2] <html>                                                                                              
##  [3] <body>                                                                                              
##  [4]                                                                                                     
##  [5] <pre>#  Last update 17-Jul-2019 by E. Di Lorenzo                                                    
##  [6] #  NPGO index monthly averages                                                                      
##  [7] #  from Jan-1950  to  Jul-2019                                                                      
##  [8] #                                                                                                   
##  [9] #  WARNING: Values after Dec-2004 are updated                                                       
## [10] #  using Satellite SSHa from AVISO Delayed Time product.                                            
## [11] #  http://opendap.aviso.oceanobs.com/thredds/dodsC/dataset-duacs-dt-global-allsat-msla-h            
## [12] #                                                                                                   
## [13] #  PRELIMINARY: Values after Jan-2019 are preliminary and updated                                   
## [14] #  using Satellite SSHa from AVISO Near Real Time product.                                          
## [15] #  http://opendap.aviso.oceanobs.com/thredds/dodsC/dataset-duacs-nrt-over30d-global-allsat-msla-h   
## [16] #                                                                                                   
## [17] #  The update is performed by taking the NPGO spatial pattern of Di Lorenzo et al. 2008             
## [18] #  computed over the period 1950-2004, and projecting the AVISO Satellite SSHa.                     
## [19] #  During the pre-processing of the AVISO data, we remove the seasonal cycle based on               
## [20] #  the 1993-2004 seasonal means.                                                                    
## [21] #                                                                                                   
## [22] #  AVISO PRODUCT UPDATE Summer 2014: AVISO has released a re-processed dataset for the sea level.   
## [23] #  Starting from the November 2014, the NPGO index is computed with this updated dataset. NPGO      
## [24] #  values from 2004 onward have been recomputed with very minor differences from previous releases. 
## [25] #                                                                                                   
## [26] #  Ref:                                                                                             
## [27] #  Di Lorenzo et al., 2008: North Pacific Gyre Oscillation                                          
## [28] #  links ocean climate and ecosystem change, GRL.                                                   
## [29] #                                                                                                   
## [30] #      YEAR            MONTH        NPGO index
\end{verbatim}

Next, we extract the actual NPGO indices for the years of interest and
inspect the file contents. We also want the average NPGO annual index
from January 1 through December 31 during the first year that the
juvenile steelhead are in the ocean (i.e., during their second year of
life). Therefore, we need NPGO values from
\texttt{yr\_frst\ +\ marine\_lag\ ==\ 1981} through
\texttt{yr\_last\ -\ age\_min\ +\ n\_fore\ +\ marine\_lag\ ==\ 2016}.

\begin{Shaded}
\begin{Highlighting}[]
\NormalTok{## number of years of data}
\NormalTok{n_yrs <-}\StringTok{ }\NormalTok{yr_last }\OperatorTok{-}\StringTok{ }\NormalTok{yr_frst }\OperatorTok{+}\StringTok{ }\DecValTok{1}
\NormalTok{## NPGO data for years of interest}
\NormalTok{dat_NPGO <-}\StringTok{ }\KeywordTok{read_table}\NormalTok{(url_NPGO, }\DataTypeTok{col_names =} \OtherTok{FALSE}\NormalTok{,}
                       \DataTypeTok{skip =}\NormalTok{ hdr_NPGO }\OperatorTok{+}\StringTok{ }\NormalTok{(yr_frst}\OperatorTok{-}\DecValTok{1950}\NormalTok{)}\OperatorTok{*}\DecValTok{12}\NormalTok{,}
                       \DataTypeTok{n_max =}\NormalTok{ (n_yrs}\OperatorTok{-}\DecValTok{1}\NormalTok{)}\OperatorTok{*}\DecValTok{12}\NormalTok{)}
\KeywordTok{colnames}\NormalTok{(dat_NPGO) <-}\StringTok{ }\KeywordTok{c}\NormalTok{(}\StringTok{"year"}\NormalTok{,}\StringTok{"month"}\NormalTok{,}\StringTok{"NPGO"}\NormalTok{)}
\NormalTok{## select only years of interest indexed by brood year }
\NormalTok{dat_NPGO_wtr <-}\StringTok{ }\KeywordTok{subset}\NormalTok{(dat_NPGO, (month }\OperatorTok{==}\StringTok{ }\DecValTok{12}\NormalTok{)}
                   \OperatorTok{&}\StringTok{ }\NormalTok{year }\OperatorTok{>=}\StringTok{ }\NormalTok{yr_frst}
                   \OperatorTok{&}\StringTok{ }\NormalTok{year }\OperatorTok{<=}\StringTok{ }\NormalTok{yr_last}\OperatorTok{-}\NormalTok{age_min}\OperatorTok{+}\NormalTok{n_fore)}

\NormalTok{dat_NPGO_spr <-}\StringTok{ }\KeywordTok{subset}\NormalTok{(dat_NPGO, (month }\OperatorTok{>=}\StringTok{ }\DecValTok{1} \OperatorTok{&}\StringTok{ }\NormalTok{month }\OperatorTok{<=}\StringTok{ }\DecValTok{3}\NormalTok{)}
                   \OperatorTok{&}\StringTok{ }\NormalTok{year }\OperatorTok{>=}\StringTok{ }\NormalTok{yr_frst}\OperatorTok{+}\NormalTok{marine_lag}
                   \OperatorTok{&}\StringTok{ }\NormalTok{year }\OperatorTok{<=}\StringTok{ }\NormalTok{yr_last}\OperatorTok{-}\NormalTok{age_min}\OperatorTok{+}\NormalTok{n_fore}\OperatorTok{+}\NormalTok{marine_lag)}

\NormalTok{## change spr year index to match wtr}
\NormalTok{dat_NPGO_spr[,}\StringTok{"year"}\NormalTok{] <-}\StringTok{ }\NormalTok{dat_NPGO_spr[,}\StringTok{"year"}\NormalTok{] }\OperatorTok{-}\StringTok{ }\NormalTok{marine_lag}

\NormalTok{## combined NPGO indexed to brood year & calculate december - March average}
\NormalTok{dat_NPGO <-}\StringTok{ }\KeywordTok{aggregate}\NormalTok{(NPGO}\OperatorTok{~}\NormalTok{year, }\DataTypeTok{data =} \KeywordTok{rbind}\NormalTok{(dat_NPGO_wtr,dat_NPGO_spr), mean)}
\NormalTok{dat_NPGO <-}\StringTok{ }\KeywordTok{data.frame}\NormalTok{(}\DataTypeTok{year =} \KeywordTok{seq}\NormalTok{(yr_frst,yr_last}\OperatorTok{-}\NormalTok{age_min}\OperatorTok{+}\NormalTok{n_fore),}
                       \DataTypeTok{NPGO =}\NormalTok{ dat_NPGO[,}\DecValTok{2}\NormalTok{])}
\NormalTok{dat_NPGO[,}\StringTok{"NPGO"}\NormalTok{] <-}\StringTok{ }\KeywordTok{round}\NormalTok{(dat_NPGO[,}\StringTok{"NPGO"}\NormalTok{], }\DecValTok{2}\NormalTok{)}
\end{Highlighting}
\end{Shaded}

\subsection{Hatchery Smolt to adult survival
rates}\label{hatchery-smolt-to-adult-survival-rates}

We used a time series of marine survival of hatchery Chum salmon from
the Tulalip Hatchery as an indicator of marine survival for conspecific
wild Chum salmon from the Skagit River.

\begin{Shaded}
\begin{Highlighting}[]
\NormalTok{dat_SAR <-}\StringTok{ }\KeywordTok{read_csv}\NormalTok{(}\KeywordTok{file.path}\NormalTok{(datadir,}\StringTok{"ps_hatchery_chum_return_rates.csv"}\NormalTok{))}
\NormalTok{dat_SAR <-}\StringTok{ }\KeywordTok{subset}\NormalTok{(dat_SAR, year }\OperatorTok{>=}\StringTok{ }\NormalTok{yr_frst}
                   \OperatorTok{&}\StringTok{ }\NormalTok{year }\OperatorTok{<=}\StringTok{ }\NormalTok{yr_last}\OperatorTok{-}\NormalTok{age_min}\OperatorTok{+}\NormalTok{n_fore)}

\NormalTok{dat_SAR <-}\StringTok{ }\KeywordTok{data.frame}\NormalTok{(dat_SAR)}

\NormalTok{dat_SAR <-}\StringTok{ }\NormalTok{dat_SAR[,}\KeywordTok{c}\NormalTok{(}\DecValTok{1}\NormalTok{,}\DecValTok{4}\NormalTok{)]}
\KeywordTok{print}\NormalTok{(dat_SAR)}
\end{Highlighting}
\end{Shaded}

\begin{verbatim}
##    year    Tulalip
## 1  1980  3.3879987
## 2  1981  5.6751522
## 3  1982 14.4417349
## 4  1983  2.1223123
## 5  1984 11.0153670
## 6  1985  8.7570109
## 7  1986 10.4798662
## 8  1987  4.0466845
## 9  1988  3.0621649
## 10 1989  2.2609455
## 11 1990  9.9789032
## 12 1991  1.3910264
## 13 1992  3.6152615
## 14 1993  2.2592288
## 15 1994  2.2719406
## 16 1995  4.4830040
## 17 1996  1.1566535
## 18 1997  5.3983225
## 19 1998 14.6207124
## 20 1999  6.7131602
## 21 2000  3.7462885
## 22 2001  5.1323788
## 23 2002  2.6584741
## 24 2003  2.7421580
## 25 2004  6.6129158
## 26 2005  3.7972014
## 27 2006  2.0014710
## 28 2007  3.5351005
## 29 2008  0.6264500
## 30 2009  4.0434074
## 31 2010  1.7306812
## 32 2011  0.3638305
\end{verbatim}

\subsection{Pink salmon escapement}\label{pink-salmon-escapement}

We used pink salmon escapement as a covariate in the model. The
plausible hypothesis is that pink salmon fry compete with Chum fry for
food and space and therefore may reduce the overall prodcutivity of Chum
salmon.

\begin{Shaded}
\begin{Highlighting}[]
\NormalTok{dat_pink_esc <-}\StringTok{ }\KeywordTok{read_csv}\NormalTok{(}\KeywordTok{file.path}\NormalTok{(datadir,}\StringTok{"skagit_pink_esc.csv"}\NormalTok{))}
\NormalTok{dat_pink_esc <-}\StringTok{ }\KeywordTok{subset}\NormalTok{(dat_pink_esc, year }\OperatorTok{>=}\StringTok{ }\NormalTok{yr_frst}
                   \OperatorTok{&}\StringTok{ }\NormalTok{year }\OperatorTok{<=}\StringTok{ }\NormalTok{yr_last}\OperatorTok{-}\NormalTok{age_min}\OperatorTok{+}\NormalTok{n_fore)}
\NormalTok{dat_pink_esc <-}\StringTok{ }\KeywordTok{data.frame}\NormalTok{(dat_pink_esc[,}\KeywordTok{c}\NormalTok{(}\DecValTok{1}\NormalTok{,}\DecValTok{3}\NormalTok{)])}

\KeywordTok{print}\NormalTok{(dat_pink_esc)}
\end{Highlighting}
\end{Shaded}

\begin{verbatim}
##    year whidbey_basin_pink_escapement
## 1  1980                             0
## 2  1981                        208728
## 3  1982                             0
## 4  1983                        794922
## 5  1984                             0
## 6  1985                       1212444
## 7  1986                             0
## 8  1987                        866906
## 9  1988                             0
## 10 1989                        551870
## 11 1990                             0
## 12 1991                        611447
## 13 1992                             0
## 14 1993                        740135
## 15 1994                             0
## 16 1995                       1166626
## 17 1996                             0
## 18 1997                        252109
## 19 1998                             0
## 20 1999                        781543
## 21 2000                             0
## 22 2001                       2741709
## 23 2002                             0
## 24 2003                       2144081
## 25 2004                             0
## 26 2005                        660124
## 27 2006                             0
## 28 2007                       1683591
## 29 2008                             0
## 30 2009                       3992373
## 31 2010                             0
## 32 2011                       1172903
## 33 2012                             0
## 34 2013                       3053569
## 35 2014                             0
## 36 2015                        770674
\end{verbatim}

\begin{Shaded}
\begin{Highlighting}[]
\NormalTok{dat_pink_esc <-}\StringTok{ }\NormalTok{dat_pink_esc[}\KeywordTok{which}\NormalTok{(dat_pink_esc}\OperatorTok{$}\NormalTok{year }\OperatorTok\StringTok{ }\KeywordTok{seq}\NormalTok{(yr_frst,yr_last,}\DecValTok{1}\NormalTok{)),]}
\KeywordTok{print}\NormalTok{(dat_pink_esc)}
\end{Highlighting}
\end{Shaded}

\begin{verbatim}
##    year whidbey_basin_pink_escapement
## 1  1980                             0
## 2  1981                        208728
## 3  1982                             0
## 4  1983                        794922
## 5  1984                             0
## 6  1985                       1212444
## 7  1986                             0
## 8  1987                        866906
## 9  1988                             0
## 10 1989                        551870
## 11 1990                             0
## 12 1991                        611447
## 13 1992                             0
## 14 1993                        740135
## 15 1994                             0
## 16 1995                       1166626
## 17 1996                             0
## 18 1997                        252109
## 19 1998                             0
## 20 1999                        781543
## 21 2000                             0
## 22 2001                       2741709
## 23 2002                             0
## 24 2003                       2144081
## 25 2004                             0
## 26 2005                        660124
## 27 2006                             0
## 28 2007                       1683591
## 29 2008                             0
## 30 2009                       3992373
## 31 2010                             0
## 32 2011                       1172903
## 33 2012                             0
## 34 2013                       3053569
## 35 2014                             0
## 36 2015                        770674
\end{verbatim}

\section{Archive covariates}\label{archive-covariates}

The last thing we will do is combine the covariates into one data frame
and write them to a file for use in the analysis.

\begin{Shaded}
\begin{Highlighting}[]
\CommentTok{#print(dat_flow_wtr)}
\CommentTok{#print(dat_flow_spr)}
\NormalTok{## combine covariates}
\NormalTok{dat_cvrs <-}\StringTok{ }\KeywordTok{Reduce}\NormalTok{(}\ControlFlowTok{function}\NormalTok{(...) }\KeywordTok{merge}\NormalTok{(..., }\DataTypeTok{all =} \OtherTok{TRUE}\NormalTok{),}
                   \KeywordTok{list}\NormalTok{(dat_flow_wtr,}
\NormalTok{                        dat_flow_spr,dat_NPGO,}
\NormalTok{                        dat_pink_esc))}

\NormalTok{## check table of covariates}
\KeywordTok{print}\NormalTok{(dat_cvrs)}
\end{Highlighting}
\end{Shaded}

\begin{verbatim}
##    year flow_wtr flow_spr  NPGO whidbey_basin_pink_escapement
## 1  1980   2860.0   917.46 -0.57                             0
## 2  1981   1526.3  1347.88 -0.29                        208728
## 3  1982   1789.6   846.67  0.32                             0
## 4  1983   2302.2   923.13  0.06                        794922
## 5  1984    775.9  1175.15  0.19                             0
## 6  1985   1795.3  1030.73 -1.09                       1212444
## 7  1986   1843.4  1118.51  0.32                             0
## 8  1987    909.0  1095.86  0.89                        866906
## 9  1988   1336.6   846.67  0.91                             0
## 10 1989   2497.5   841.01  0.12                        551870
## 11 1990   4021.0   869.33 -0.85                             0
## 12 1991   1135.5  1160.99 -0.25                        611447
## 13 1992    781.5  1022.24 -2.14                             0
## 14 1993   1030.7  1030.73 -1.92                        740135
## 15 1994   1577.2   736.24 -1.28                             0
## 16 1995   3737.8   906.14 -0.66                       1166626
## 17 1996   2089.8  2089.78 -1.18                             0
## 18 1997   1008.1   682.44  0.90                        252109
## 19 1998   1469.6  1197.80  1.53                             0
## 20 1999   2174.7  1033.56  1.68                        781543
## 21 2000    546.5   744.73  2.59                             0
## 22 2001   2086.9  1452.65  1.99                       2741709
## 23 2002   1500.8  1279.92  1.77                             0
## 24 2003   1826.4   753.23  0.38                       2144081
## 25 2004   1891.6   569.17 -1.41                             0
## 26 2005   1625.4  1095.86 -0.92                        660124
## 27 2006   3539.6  2101.11 -0.31                             0
## 28 2007   2019.0  1735.82  0.88                       1683591
## 29 2008   2064.3   886.32  0.60                             0
## 30 2009   1868.9   843.84  1.66                       3992373
## 31 2010   2364.5  1177.98  0.71                             0
## 32 2011   1619.7  1628.22  0.74                       1172903
## 33 2012   1141.2  1325.23  1.07                             0
## 34 2013   1381.9  1381.86 -0.64                       3053569
## 35 2014   2418.3   673.94 -0.89                             0
## 36 2015   2070.0   761.72  0.13                        770674
\end{verbatim}

\begin{Shaded}
\begin{Highlighting}[]
\NormalTok{## write covariates to a file}
\KeywordTok{write_csv}\NormalTok{(dat_cvrs, }\KeywordTok{file.path}\NormalTok{(datadir, }\StringTok{"skagit_chum_covars.csv"}\NormalTok{))}
\end{Highlighting}
\end{Shaded}


\end{document}
